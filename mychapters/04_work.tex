%----------------------------------------------------------------------------
\chapter{Szimulációs eszköz tervezése és megvalósítása}
%----------------------------------------------------------------------------
Ebben a fejezetben kifejezette a tervezés lépéseit és a munka logika felépítését szeretném bemutatni. A cél egy olyan rendszer létrehozása volt ami minimális felhasználói beavatkozásra képes elindítani egy szimulációs környezetet a robotról és méréseket futattni.

\section{Probléma felvetés}
Az optimlizációs algoritmusokra és magvalósításukra a későbbiekben fogok kitérni. Jelen fejezet kontextusában az optimalizáció mint matematika folyamatból lényeges, hogy szükség van a controller működése közben mérhető adatokra, melyekből aztán az optimalizáció során egy minimum vagy maximum-ot kereső algoritmushoz előállítható egy score függvény. Egy olyan mérések során